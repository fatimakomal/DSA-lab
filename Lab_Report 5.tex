\documentclass[11pt]{article}            % Report class in 11 points
\parindent0pt  \parskip10pt             % make block paragraphs
\usepackage{graphicx}
\usepackage{listings}
\graphicspath{ {images/} }
\usepackage{graphicx} %  graphics header file
\begin{document}
\begin{titlepage}
    \centering
  \vfill
    \includegraphics[width=8cm]{uni_logo.png} \\ 
	\vskip2cm
    {\bfseries\Large
	Data Structuers and algorithms \\ (CS09203)\\
	
	\vskip2cm
	Lab Report 
	 
	\vskip2cm
	}    

\begin{center}
\begin{tabular}{ l l  } 

Name: & Fatima komal \\ 
Registration \#: & SEU-F16-143 \\ 
Lab Report \#: & 05 \\ 
 Dated:& 16-04-2018\\ 
Submitted To:& Mr. Usman Ahmed\\ 

 %\hline
\end{tabular}
\end{center}
    \vfill
    The University of Lahore, Islamabad Campus\\
Department of Computer Science \& Information Technology
\end{titlepage}


    
    {\bfseries\Large
\centering
	Experiment \# 5 \\

Link list -Basic Deletion at desired  Aposition l \\
	
	}    
 \vskip1cm
 \textbf {Objective}\\ The objective of this session is to insertion, traversal and deletion at desired position in link list 
using C++. 
 
 \textbf {Software Tool} \\
 1. Code Blocks with GCC compiler.

\section{Theory }              

This section discusses how to insert an item into, and delete an item from, a linked list. 
Consider the following definition of a node. (For simplicity, we assume that the info type is int. 
 
  struct nodeType 
 {  
int info nodeType* link; 
  }; 


\section{Task}  

\subsection{ Task 1 }     
\begin{figure*}
\centering
  \includegraphics[width=12cm,height=6cm,keepaspectratio]{6.png}
\caption{output}
\label{Figure:3}    
\end{figure*}

Write a C++ code using functions for the following operations. 
 
1. 
Creating a linked List. 
2. 
Traversing a Linked List. 
3. 
Inserting the node at the start of the list. 
4. 
Inserting a node after a given node. 
5. 
Inserting a node in a sorted list. 
Create a complete menu for the above options and also create option for reusing 
it. 
 

\subsection{Procedure: Task 1 }    
 

\begin{lstlisting}

#include<iostream>
#include<stdlib.h>
using namespace std;
struct Node{
	int data;
	struct Node*next;
	
};
struct Node*head=NULL;
void insert(int x){
	struct Node*temp=(Node*)malloc(sizeof(struct Node));
	temp->data=x;
	temp->next=head;
	head=temp;
}

void print()
{
	struct Node*temp=head;
	cout<<"list is ";
	while(temp!=NULL)
	{
		cout<<temp->data<<" ";
		temp=temp->next;
		
	}
	cout<<endl;
}
 void Delete(int n)
{
	struct Node*temp1=head;
	if(n==1){
		head=temp1->next;
		free(temp1);
		return;
		
	}
	int i;
	for(i=0;i<n-1;i++){
		temp1=temp1->next;
		struct Node*temp2=temp1->next;
		temp1->next=temp2->next;
		free(temp2);
	}
}
int main(){
	head=NULL;
	cout<<"how many numbers"<<endl;
	int n,i,x,y;
	cin>>n;
	for(i=0;i<n;i++){
		cout<<"enter the number"<<endl;
		cin>>x;
		insert(x);
		print();
	
	}
	
		cout<<"enter want you to delete"<<endl;
		cin>>y;
		Delete(y);
		print();
	
}
\end{lstlisting}


 
\end{document}                          % The required last line
