\documentclass[11pt]{article}            % Report class in 11 points
\parindent0pt  \parskip10pt             % make block paragraphs
\usepackage{graphicx}
\usepackage{listings}
\graphicspath{ {images/} }
\usepackage{graphicx} %  graphics header file
\begin{document}
\begin{titlepage}
    \centering
  \vfill
    \includegraphics[width=8cm]{uni_logo.png} \\ 
	\vskip2cm
    {\bfseries\Large
	Data Structuers and algorithms \\ (CS09203)\\
	
	\vskip2cm
	Lab Report 
	 
	\vskip2cm
	}    

\begin{center}
\begin{tabular}{ l l  } 

Name: & Fatima komal \\ 
Registration \#: & SEU-F16-143 \\ 
Lab Report \#: & 11 \\ 
 Dated:& 27-06-2018\\ 
Submitted To:& Mr. Usman Ahmed\\ 

 %\hline
\end{tabular}
\end{center}
    \vfill
    The University of Lahore, Islamabad Campus\\
Department of Computer Science \& Information Technology
\end{titlepage}


    
    {\bfseries\Large
\centering
	Experiment \# 11 \\

Kruskal's algorithml \\
	
	}    
 \vskip1cm
 \textbf {Objective}\\ The objective of this session is to show the representation of trees using C++. 
 
 \textbf {Software Tool} \\
 1. Code Blocks with GCC compiler.

\section{Theory }              

Kruskal's algorithm is a minimum-spanning-tree algorithm which finds an edge of the least possible weight that connects any two trees in the forest.[1] It is a greedy algorithm in graph theory as it finds a minimum spanning tree for a connected weighted graph adding increasing cost arcs at each step.[1] This means it finds a subset of the edges that forms a tree that includes every vertex, where the total weight of all the edges in the tree is minimized. If the graph is not connected, then it finds a minimum spanning forest (a minimum spanning tree for each connected component).
\section{Task}  

\subsection{ Task 1 }     

Impement Kruskal's algorithm.

\subsection{Procedure: Task 1 }    
 

\begin{lstlisting}


#include<bits/stdc++.h>
using namespace std;
 
// Creating shortcut for an integer pair
typedef  pair<int, int> iPair;
 
// Structure to represent a graph
struct Graph
{
    int V, E;
    vector< pair<int, iPair> > edges;
 
    // Constructor
    Graph(int V, int E)
    {
        this->V = V;
        this->E = E;
    }
 
    // Utility function to add an edge
    void addEdge(int u, int v, int w)
    {
        edges.push_back({w, {u, v}});
    }
 
    // Function to find MST using Kruskal's
    // MST algorithm
    int kruskalMST();
};
 
// To represent Disjoint Sets
struct DisjointSets
{
    int *parent, *rnk;
    int n;
 
    // Constructor.
    DisjointSets(int n)
    {
        // Allocate memory
        this->n = n;
        parent = new int[n+1];
        rnk = new int[n+1];
 
        // Initially, all vertices are in
        // different sets and have rank 0.
        for (int i = 0; i <= n; i++)
        {
            rnk[i] = 0;
 
            //every element is parent of itself
            parent[i] = i;
        }
    }
 
    // Find the parent of a node 'u'
    // Path Compression
    int find(int u)
    {
        /* Make the parent of the nodes in the path
           from u--> parent[u] point to parent[u] */
        if (u != parent[u])
            parent[u] = find(parent[u]);
        return parent[u];
    }
 
    // Union by rank
    void merge(int x, int y)
    {
        x = find(x), y = find(y);
 
        /* Make tree with smaller height
           a subtree of the other tree  */
        if (rnk[x] > rnk[y])
            parent[y] = x;
        else // If rnk[x] <= rnk[y]
            parent[x] = y;
 
        if (rnk[x] == rnk[y])
            rnk[y]++;
    }
};
 
 /* Functions returns weight of the MST*/
 
int Graph::kruskalMST()
{
    int mst_wt = 0; // Initialize result
 
    // Sort edges in increasing order on basis of cost
    sort(edges.begin(), edges.end());
 
    // Create disjoint sets
    DisjointSets ds(V);
 
    // Iterate through all sorted edges
    vector< pair<int, iPair> >::iterator it;
    for (it=edges.begin(); it!=edges.end(); it++)
    {
        int u = it->second.first;
        int v = it->second.second;
 
        int set_u = ds.find(u);
        int set_v = ds.find(v);
 
        // Check if the selected edge is creating
        // a cycle or not (Cycle is created if u
        // and v belong to same set)
        if (set_u != set_v)
        {
            // Current edge will be in the MST
            // so print it
            cout << u << " - " << v << endl;
 
            // Update MST weight
            mst_wt += it->first;
 
            // Merge two sets
            ds.merge(set_u, set_v);
        }
    }
 
    return mst_wt;
}
 
// Driver program to test above functions
int main()
{
    /* Let us create above shown weighted
       and unidrected graph */
    int V = 9, E = 14;
    Graph g(V, E);
 
    //  making above shown graph
    g.addEdge(0, 1, 4);
    g.addEdge(0, 7, 8);
    g.addEdge(1, 2, 8);
    g.addEdge(1, 8,5 );
    g.addEdge(1, 6, 10);
    g.addEdge(2, 6, 4);
    g.addEdge(2, 3, 4);
    g.addEdge(2, 8, 4);
    g.addEdge(2, 5, 4);
    g.addEdge(2, 1, 8);
    g.addEdge(3, 6, 3);
    g.addEdge(3, 2, 4);
    g.addEdge(3, 4, 3);
    g.addEdge(4, 3, 3);
    
    g.addEdge(4, 6, 6);
    g.addEdge(4, 5, 1);
    g.addEdge(4, 7, 2);
    g.addEdge(5, 2, 4);
    g.addEdge(5, 7, 3);
    g.addEdge(5, 4, 1);
 
    g.addEdge(6, 1, 10);
    g.addEdge(6, 2, 4);
    g.addEdge(6, 3, 3);
    g.addEdge(6, 4, 6);
    
    g.addEdge(7, 4, 2);
    g.addEdge(7, 5, 3);
    g.addEdge(7, 8, 3);
    g.addEdge(8, 1, 5);
    g.addEdge(8, 2, 4);
    g.addEdge(8, 5, 3);
    cout << "Edges of MST are \n";
    int mst_wt = g.kruskalMST();
 
    cout << "\nWeight of MST is " << mst_wt;
 
    return 0;
}





\end{lstlisting}
\begin{figure*}
\centering
  \includegraphics[width=12cm,height=6cm,keepaspectratio]{11.png}
\caption{output}
\label{Figure:3}    
\end{figure*} 




 
\end{document}                          % The required last line
