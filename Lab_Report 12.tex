\documentclass[11pt]{article}            % Report class in 11 points
\parindent0pt  \parskip10pt             % make block paragraphs
\usepackage{graphicx}
\usepackage{listings}
\graphicspath{ {images/} }
\usepackage{graphicx} %  graphics header file
\begin{document}
\begin{titlepage}
    \centering
  \vfill
    \includegraphics[width=8cm]{uni_logo.png} \\ 
	\vskip2cm
    {\bfseries\Large
	Data Structuers and algorithms \\ (CS09203)\\
	
	\vskip2cm
	Lab Report 
	 
	\vskip2cm
	}    

\begin{center}
\begin{tabular}{ l l  } 

Name: & Fatima komal \\ 
Registration \#: & SEU-F16-143 \\ 
Lab Report \#: & 12 \\ 
 Dated:& 27-06-2018\\ 
Submitted To:& Mr. Usman Ahmed\\ 

 %\hline
\end{tabular}
\end{center}
    \vfill
    The University of Lahore, Islamabad Campus\\
Department of Computer Science \& Information Technology
\end{titlepage}


    
    {\bfseries\Large
\centering
	Experiment \# 12 \\

prim's algorithml \\
	
	}    
 \vskip1cm
 \textbf {Objective}\\ The objective of this session is to show the representation of trees using C++. 
 
 \textbf {Software Tool} \\
 1. Code Blocks with GCC compiler.

\section{Theory }              

 Prim's algorithm is a greedy algorithm that finds a minimum spanning tree for a weighted undirected graph. This means it finds a subset of the edges that forms a tree that includes every vertex, where the total weight of all the edges in the tree is minimized. The algorithm operates by building this tree one vertex at a time, from an arbitrary starting vertex, at each step adding the cheapest possible connection from the tree to another vertex.
\section{Task}  

\subsection{ Task 1 }     

Impement  Prim's algorithm.

\subsection{Procedure: Task 1 }    
 

\begin{lstlisting}




// A C / C++ program for Prim's Minimum Spanning Tree (MST) algorithm. 
// The program is for adjacency matrix representation of the graph
 
#include <stdio.h>
#include <limits.h>
 using namespace std;
// Number of vertices in the graph
#define V 8
 
// A utility function to find the vertex with minimum key value, from
// the set of vertices not yet included in MST
int minKey(int key[], bool mstSet[])
{
   // Initialize min value
   int min = INT_MAX, min_index;
 
   for (int v = 0; v < V; v++)
     if (mstSet[v] == false && key[v] < min)
         min = key[v], min_index = v;
 
   return min_index;
}
 
// A utility function to print the constructed MST stored in parent[]
int printMST(int parent[], int n, int graph[V][V])
{
   printf("Edge   Weight\n");
   for (int i = 1; i < V; i++)
      printf("%d - %d    %d \n", parent[i], i, graph[i][parent[i]]);
}
 
// Function to construct and print MST for a graph represented using adjacency
// matrix representation
void primMST(int graph[V][V])
{
     int parent[V]; // Array to store constructed MST
     int key[V];   // Key values used to pick minimum weight edge in cut
     bool mstSet[V];  // To represent set of vertices not yet included in MST
 
     // Initialize all keys as INFINITE
     for (int i = 0; i < V; i++)
        key[i] = INT_MAX, mstSet[i] = false;
 
     // Always include first 1st vertex in MST.
     key[0] = 0;     // Make key 0 so that this vertex is picked as first vertex
     parent[0] = -1; // First node is always root of MST 
 
     // The MST will have V vertices
     for (int count = 0; count < V-1; count++)
     {
        // Pick the minimum key vertex from the set of vertices
        // not yet included in MST
        int u = minKey(key, mstSet);
 
        // Add the picked vertex to the MST Set
        mstSet[u] = true;
 
        // Update key value and parent index of the adjacent vertices of
        // the picked vertex. Consider only those vertices which are not yet
        // included in MST
        for (int v = 0; v < V; v++)
 
           // graph[u][v] is non zero only for adjacent vertices of m
           // mstSet[v] is false for vertices not yet included in MST
           // Update the key only if graph[u][v] is smaller than key[v]
          if (graph[u][v] && mstSet[v] == false && graph[u][v] <  key[v])
             parent[v]  = u, key[v] = graph[u][v];
     }
 
     // print the constructed MST
     printMST(parent, V, graph);
}
 
 
// driver program to test above function
int main()
{
   
   int graph[V][V] = {{1, 8, 0, 0, 0,10,0,5},
                      {8, 0, 4, 0, 4,4,0,4},
                      {0, 4, 0, 3, 0,3,0,0},
                      {0, 0, 3, 0, 1,6,2,0},
                      {0, 4, 0, 1, 0,0,3,0},
	                 {10, 4, 3, 6, 0,0,0,0},
                      {0, 0, 0, 2, 3,0,0,3},
                      {5, 4, 0, 0, 0,0,3,0},

          
		  
		  
		             };
 
    // Print the solution
    primMST(graph);
 
    return 0;
}







\end{lstlisting}
\begin{figure*}
\centering
  \includegraphics[width=12cm,height=6cm,keepaspectratio]{12.png}
\caption{output}
\label{Figure:3}    
\end{figure*} 




 
\end{document}                          % The required last line
