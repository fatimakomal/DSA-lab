\documentclass[11pt]{article}            % Report class in 11 points
\parindent0pt  \parskip10pt             % make block paragraphs
\usepackage{graphicx}
\usepackage{listings}
\graphicspath{ {images/} }
\usepackage{graphicx} %  graphics header file
\begin{document}
\begin{titlepage}
    \centering
  \vfill
    \includegraphics[width=8cm]{uni_logo.png} \\ 
	\vskip2cm
    {\bfseries\Large
	Data Structuers and algorithms \\ (CS09203)\\
	
	\vskip2cm
	Lab Report 
	 
	\vskip2cm
	}    

\begin{center}
\begin{tabular}{ l l  } 

Name: & Fatima komal \\ 
Registration \#: & SEU-F16-143 \\ 
Lab Report \#: & 01 \\ 
 Dated:& 16-04-2018\\ 
Submitted To:& Mr. Usman Ahmed\\ 

 %\hline
\end{tabular}
\end{center}
    \vfill
    The University of Lahore, Islamabad Campus\\
Department of Computer Science \& Information Technology
\end{titlepage}


    
    {\bfseries\Large
\centering
	Experiment \# 2 \\

Queue with Array implementation  \\
	
	}    
 \vskip1cm
 \textbf {Objective}\\The  objective  of  this  session  is  to  understand  the  various  operations  on  queues  using  array 
structure in C++.
 \textbf {Software Tool} \\
 1.To achieve the goals of objectives, I use Code Blocks with GCC compiler

\section{Theory }              

Queue using Array: - 
 
This  manual  discusses  an  important  data  structure,  called  a  queue.  The  idea of a  queue  in 
computer  science  is  the  same  as  the  idea  of  the  queues  to  which  you  are  accustomed  in 
ever yday life. There are queues of customers in a bank or in a grocery store and queues of cars 
waiting to pass through a tollbooth. Similarly, because a computer can send a pr int request faster 
than a printer can print, a queue of documents is often waiting to be printed at a printer. The 
general rule to process elements in a queue is that the customer at the front of the queue is served 
next and that when a new customer arr ives, he or she stands at the end of the queue. That is, a 
queue is a First In Fir st Out data structure. 
 
A queue is a set of elements of the same type in which the elements are added at one end, called 
the back or rear, and deleted from the other end, called the front. For example, consider a line 
of  customers  in a  bank,  wherein  the  customers are waiting  to  withdraw/deposit  money or to 
conduct some other business. Each new customer gets in the line at the rear. Whenever a teller 
is ready for a new customer, the customer at the front of the line is served. 
 
The rear of the queue is accessed whenever a new element is added to the queue, and the front 
of  the queue is  accessed whenever an element  is deleted  from the queue.  As  in a  stack, the 
middle elements of the queue are inaccessible, even if the queue elements are stored in an array. 
 
Queue: A data structure in which the elements are added at one end, called the rear, and deleted 
from the other end, called the front; a First-In-First-Out (FIFO) data structure. 
 


\section{Task}  
\subsection{ Task 1 }     

\begin{figure*}
\centering
  \includegraphics[width=12cm,height=6cm,keepaspectratio]{picture3.png}
\caption{queue}
\label{Figure:3}    
\end{figure*}
Write  a  C++  code  to  perform  insertion  and  deletion  in  queue using arrays applying 
the algorithms given in the manual. Creat e a menu .re n is the number of disks.

\subsection{Procedure: Task 1 }     

\begin{lstlisting}
#include<iostream>
using namespace std;

int main(){
	int a=-1,y=0;
	int array[100];
	char op;
	cout<<"To push enter p";
	cout<<"\nTo delete enter d";
	cout<<"\nTo display enter s";
	cout<<"\nTo exit enter e\n";
	line:
	cin>>op;
	switch (op){
	
	case 'p':
		cout<<"enter no to push\n";
		a++;
		cin>>array[a];
		cout<<"pushed at "<<a<<"\n";
		break;
	case 'd':
		cout<<"dlting \n";
		y++;
		break;
	case 's':
		for(int b=y;b<=a;b++){
			cout<<array[b]<<endl;
		}
		break;
		case 'e':
			exit;
}
goto line;
}


\end{lstlisting}

 
\end{document}                          % The required last line
