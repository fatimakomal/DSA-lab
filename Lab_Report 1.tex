\documentclass[11pt]{article}            % Report class in 11 points
\parindent0pt  \parskip10pt             % make block paragraphs
\usepackage{graphicx}
\usepackage{listings}
\graphicspath{ {images/} }
\usepackage{graphicx} %  graphics header file
\begin{document}
\begin{titlepage}
    \centering
  \vfill
    \includegraphics[width=8cm]{uni_logo.png} \\ 
	\vskip2cm
    {\bfseries\Large
	Data Structuers and algorithms \\ (CS09203)\\
	
	\vskip2cm
	Lab Report 
	 
	\vskip2cm
	}    

\begin{center}
\begin{tabular}{ l l  } 

Name: & Fatima komal \\ 
Registration \#: & SEU-F16-143 \\ 
Lab Report \#: & 01 \\ 
 Dated:& 16-04-2018\\ 
Submitted To:& Mr. Usman Ahmed\\ 

 %\hline
\end{tabular}
\end{center}
    \vfill
    The University of Lahore, Islamabad Campus\\
Department of Computer Science \& Information Technology
\end{titlepage}


    
    {\bfseries\Large
\centering
	Experiment \# 1 \\

Introduction to Arrays and its operation\\
	
	}    
 \vskip1cm
 \textbf {Objective}\\ The objectives of this lab session are to understand the basic and various operations on arrays in C++.
 
 \textbf {Software Tool} \\
 1. Code Blocks with GCC compiler.

\section{Theory }              

We have already studied array in our computer programming course. We would be using the knowledge we learned there to implement different operation on arrays.

Traversing Linear Arrays:-

Let A be the collection of data elements stored in the memory of the computer. Suppose we want to print the contents of each element of A or suppose we want to count the number of elements of A with a given property. This can be accomplished by traversing A that is by accessing and Processing each element of A exactly once.

The following algorithm traverses a linear array. The simplicity of the algorithm comes from the fact that LA is a linear structure. Other linear structures such as linked list can also be easily traversed. On the other hand the traversal of non-linear structures such as trees and graphs is considerably more complicated.


\section{Task}  
\subsection{ Task 1 }     

\begin{figure*}
\centering
  \includegraphics[width=12cm,height=6cm,keepaspectratio]{5.png}
\caption{Array}
\label{Figure:3}    
\end{figure*}

\subsection{Procedure: Task 1 }     

\begin{lstlisting}

#include<iostream>
using namespace std;

int main(){
	int a=-1,y=0;
	int array[100];
	char op;
	cout<<"press T to Traversing a Linear Array";
	cout<<"\npress I to Inset an element ";

	cout<<"\nPlease enter your choice\n";
	line:
	cin>>op;
	switch (op){
	

case 'i':
		cout<<"enter element to insert\n";
		a++;
		cin>>array[a];
		cout<<"inserted at location "<<a<<"\n";
		break;
	case 't':
		cout<<"Traversing a Linear Array\n";
		for(int b=0;b<=a;b++){
			cout<<array[b]<<endl;
		}
		break;
}
goto line;
}

\end{lstlisting}

 
\end{document}                          % The required last line
