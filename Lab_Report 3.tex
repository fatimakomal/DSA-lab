\documentclass[11pt]{article}            % Report class in 11 points
\parindent0pt  \parskip10pt             % make block paragraphs
\usepackage{graphicx}
\usepackage{listings}
\graphicspath{ {images/} }
\usepackage{graphicx} %  graphics header file
\begin{document}
\begin{titlepage}
    \centering
  \vfill
    \includegraphics[width=8cm]{uni_logo.png} \\ 
	\vskip2cm
    {\bfseries\Large
	Data Structuers and algorithms \\ (CS09203)\\
	
	\vskip2cm
	Lab Report 
	 
	\vskip2cm
	}    

\begin{center}
\begin{tabular}{ l l  } 

Name: & Fatima komal \\ 
Registration \#: & SEU-F16-143 \\ 
Lab Report \#: & 01 \\ 
 Dated:& 16-04-2018\\ 
Submitted To:& Mr. Usman Ahmed\\ 

 %\hline
\end{tabular}
\end{center}
    \vfill
    The University of Lahore, Islamabad Campus\\
Department of Computer Science \& Information Technology
\end{titlepage}


    
    {\bfseries\Large
\centering
	Experiment \# 3\\

Stack wit h Array implementat ion  \\
	
	}    
 \vskip1cm\ 
 \textbf {Objective}\\The objective of this session is to understand the various oper ations on stack using arrays structure 
in C++. 
 
 \textbf {Software Tool} \\
 1.  Code Blocks with GCC compiler.    
   

\section{Theory }              

Stacks are the most important in data structures. The notation of  a stack in computer science is the 
same as the notion  of the  Stack to which you are accustomed in everyday  life. For  example, a 
recursion program on which function call itself, but what happen when a function which is calling 
itself call another function. Such as a function ‘A’ call function ‘B’ as a recursion. So, the firstly 
function ‘B’ is call in ‘A’ and then function ‘A’ is work. So, this is a Stack. This is a Stack is First 
in Last Out data structure. 
 
Insertions in Stack: 
In Stacks, we know the array work, sometimes we need to modif y it or add some element in it. For 
that purpose, we use insertion scheme. By the use of this scheme we insert any element in Stacks 
using array. In Stack, we maintain only one node which is called TOP. And Push terminology is 
used as insertions.   
 
Deletion in Stack: 
In the deletion process, the element of the Stack is deleted on the same node which is called TOP. 
In stacks, it’s just deleting the index of the TOP element which is added at last. In Stacks Pop 
terminology is used as deletion.   
 
Display of Stack: 
In displaying section, the elements of Stacks ar e being display by using loops and variables as a 
reverse order. Such that, last element is display at on first and first element enters display at on 
last. 
 


\section{Task}  
\subsection{ Task 1 }     

\begin{figure*}
\centering
  \includegraphics[width=12cm,height=6cm,keepaspectratio]{Picture4.png}
\caption{stack}
\label{Figure:3}    
\end{figure*}
1.  Insertion in stack  
2.  Deletion in stack 
3.  Display the stack 
\subsection{Procedure: Task 1 }     

\begin{lstlisting}
#include<iostream>
using namespace std;

int main(){
	int a=-1,y;
	int array[100];
	char op;
	cout<<"To push enter p";
	cout<<"\nTo delete enter d";
	cout<<"\nTo display enter s";
	cout<<"\nTo exit enter e\n";
	line:
	cin>>op;
	switch (op){
	
	case 'p':
		cout<<"enter no to push\n";
		a++;
		cin>>array[a];
		cout<<"pushed at "<<a<<"\n";
		break;
	case 'd':
		cout<<"dlting current\n";
		a--;
		break;
	case 's':
		for(int b=0;b<=a;b++){
			cout<<array[b]<<endl;
		}
		break;
		case 'e':
			exit;
}
goto line;
}


\end{lstlisting}

 
\end{document}                          % The required last line
